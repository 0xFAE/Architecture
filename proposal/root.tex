\documentclass[11pt]{article}
\usepackage[english]{babel}
\usepackage{inputenc}
\usepackage{multicol}
\usepackage{flushend}
\usepackage{fullpage}
\usepackage{epsfig}
\usepackage{caption}
\captionsetup[table]{labelsep=period}
\usepackage{multirow}
\usepackage{mathtools}
\usepackage{xcolor}
\DeclarePairedDelimiter{\ceil}{\lceil}{\rceil}

\topmargin=-0.2in
\textheight=9.5in

\pagestyle{empty}

\begin{document}

\centerline{CSCE 614 (Fall 2020) \hfill Uwacu and \textcolor{red}{lastname}}
\medskip
\centerline{\bf Computer Architecture}
\medskip

\centerline{\bf  Project proposal: }

\bigskip

\centerline{\bf Exploring Predictive Replacemebt Policies for Instruction Cache and Branch Target Buffer}

\bigskip

\centerline{\bf Diane Uwacu and Fatma \textcolor{red}{lastname}}

\bigskip

\begin{abstract}
	For our final project, we will implement a global history replacement policy for instruction cache and branch target buffer.
	The method was proven to lower instruction cache MPKI by an average of 18\% over the least recently-used policy, and it showed
	similar improvements over several other policies.
	We plan to implement and validate the results in the original work by comparing against at least one of the stated methods.
 \end{abstract}

\section{Introduction} 
The instruction cache (I-cache) stores blocks of recently used instructions, and the branch target buffer (BTB) caches targets of previously taken branches. 
This means that the I-cache improves throughput and latency while the BTB latency due to branch target re-computation, making them both essential.
The work in \cite{samira-ISCA18} explores efficient replacement policies at the I-cache and BTB levels. Authors propose a history-based algorithm that predicts
dead blocks that should be evicted with no penalty since they are guaranteed that they won't be reused before eviction.

Authors surveyed recent work on cache replacement and noticed that a common obervation in this research area is that sequences of recently accessed instructions
correlate with the likelihood of block reuse. With this knowledge, they developed the Global History Reuse Prediction (GHRP) to predict reuse behaviors in I-cache and BTB.

Authors evaluated seven recent replacement policies and discussed their effect on the I-cache and BTB hit rates and did a comparison based average MPKI values.
Authors validated their work and run comparative analysis using benchmarks from the fifth Championship Branch Prediction Competition (CBP-5) \cite{cbp-5}.
With their experimental evaluation, they were able to explain why sampling-based replacement policies failed to improve things on the I-cache and BTB level.

For our final project, we would like to implement the GHRP replacement policy and run it through the CBP-5 benchmarks. Time allowing, we would like to compare our 
results to one of the methods that the original paper compared againt.
\section{Background}
1. no much work has been done since [26]
2. what others have done and why it's not as good
3. authors discussed 7 methods 
\section{Motivation}
\section{Proposed technique}
\section{Estimated timeline}
\label{sec:related}

 
{%\scriptsize
\bibliographystyle{abbrv}
\bibliography{references}
}

\end{document}



