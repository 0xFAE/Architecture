\documentclass[11pt]{article}
\usepackage[english]{babel}
\usepackage{inputenc}
\usepackage{multicol}
\usepackage{flushend}
\usepackage{fullpage}
\usepackage{epsfig}
\usepackage{caption}
\captionsetup[table]{labelsep=period}
\usepackage{multirow}
\usepackage{mathtools}
\usepackage{xcolor}
\DeclarePairedDelimiter{\ceil}{\lceil}{\rceil}

\topmargin=-0.2in
\textheight=9.5in

\pagestyle{empty}

\begin{document}

\centerline{CSCE 614 (Fall 2020) \hfill Uwacu and \textcolor{red}{lastname}}
\medskip
\centerline{\bf Computer Architecture}
\medskip

\centerline{\bf  Project proposal: }

\bigskip

\centerline{\bf Exploring Predictive Replacemebt Policies for Instruction Cache and Branch Target Buffer}

\bigskip

\centerline{\bf Diane Uwacu and Fatma \textcolor{red}{lastname}}

\bigskip

\begin{abstract}
	For our final project, we will implement a global history replacement policy for instruction cache and branch target buffer.
	The method was proven to lower instruction cache MPKI by an average of 18\% over the least recently-used policy, and it showed
	similar improvements over several other policies.
	We plan to implement and validate the results in the original work by comparing against at least one of the stated methods.
 \end{abstract}

\section{Introduction} 
The instruction cache (I-cache) stores blocks of recently used instructions, and the branch target buffer (BTB) caches targets of previously taken branches. 
This means that the I-cache improves throughput and latency while the BTB latency due to branch target re-computation, making them both essential 
The work in \cite{samira-ISCA18} explores efficient replacement policies at the I-cache and BTB levels.
\section{Background}
\section{Motivation}
\section{Proposed technique}
\section{Estimated timeline}
\label{sec:related}

 
{%\scriptsize
\bibliographystyle{abbrv}
\bibliography{references}
}

\end{document}



